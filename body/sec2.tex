\section{proof abstract}
The proof is following \cite{jitomirskaya1999metal}
's idea.

First of all,
\begin{definition}[$g.e.v.$]
    $E$ is a generalized eigenvalue (denote as $g.e.v.$), if there exists a nonzero polynomially bounded function $\Psi(n)$ such that $H\Psi=E\Psi$. We call $\Psi(n)$ generalized eigenfunction.
\end{definition}
Then due to the fact from \cite{simon1982schrodinger} that:
\textit{spectrally almost surely,
\[
\sigma(H)=\overline{\{E:E~is~g.e.v.\}},
\]}
We only need to show:
\begin{thm}
  For a.e. $\omega$, $\forall~g.e.v.~E$, the corresponding generalized eigenfunction $\Psi_{\omega,E}(n)$ decays exponentially in $n$.
\end{thm}

Since then, $E$ is a pure point.
