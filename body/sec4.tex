\section{Proof of Theorem 2.2}
\begin{proof}
Let $\Omega_0=\Omega_1\cap\Omega_2\cap\Omega_3(\epsilon,K)$  ($\epsilon,K$ to be determined later), pick $\tilde{\omega}\in\Omega_0$, take $\tilde{E}$ a g.e.v. for $H_{\tilde{\omega}}$.
WLOG assume $\Psi(0)\neq 0$, then $\exists N_4$, s.t. $\forall n>N_4$, 0 is $(\gamma(\tilde{E})-8\epsilon_0,n,\tilde{E},\tilde{\omega})$-singular.

% Assume $2n+1$ is not eventually $(\gamma(\tilde{E})-8\epsilon_0,n,\tilde{E},\tilde{\omega})$-regular, then $\exists \{n_k\}$ with $n_k\to\infty$, s.t. $2n_k+1$ is  $(\gamma(\tilde{E})-3\epsilon_0,n_k,\tilde{E},\tilde{\omega})$-singular.

For $n>N_0=\max{N_1,N_2,N_3,N_4}$, assume $2n+1$ is $(\gamma(\tilde{E})-8\epsilon_0,n,\tilde{E},\tilde{\omega})$-singular.
% WLOG use $\{n\}$ istead of $\{n_k\}$, we have that
\begin{itemize}
  \item Both $0$ and $2n+1$ is $(\gamma(\tilde{E})-8\epsilon_0,n,\tilde{E},\tilde{\omega})$-singular.
  \item So by lemma \ref{lemma1},  $\tilde{E}\in B_{[n+1,3n+1],\epsilon_0,\tilde{\omega}}^-\cup B_{[n+1,2n+1],\epsilon_0,\tilde{\omega}}^+\cup B_{[2n+1,3n+1],\epsilon_0,\tilde{\omega}}^+$.
  \item By corollary \ref{omega2} and \eqref{B+}, $\tilde{E}\notin B_{[n+1,2n+1],\epsilon_0,\tilde{\omega}}^+\cup B_{[2n+1,3n+1],\epsilon_0,\tilde{\omega}}^+$, so it can only lies in $B_{[n+1,3n+1],\epsilon_0,\tilde{\omega}}^-$
  \item
  % But  by \eqref{B-}.
  % \begin{equation}\label{bad}
  %   B_{[n+1,3n+1],\epsilon_0,\tilde{\omega}}^-= \left\{E:|P_{[n+1,3n+1],\epsilon,E,\tilde{\omega}}|\leq e^{(\gamma(E)-\epsilon_0)(2n+1)}\right\}
  % \end{equation}
   Note that by \eqref{B-}, $P_{[n+1,3n+1],\epsilon,\epsilon_0,E,\tilde{\omega}}$ in $B=B_{[n+1,3n+1],\epsilon,\tilde{\omega}}$ is a polynomial in $E$ that have $2n+1$ real zeros (eigenvalues of $H_{[n+1,3n+1],\tilde{\omega}}$), which are all in $B$. Thus $B$ contains less than $2n+1$ intervals near the eigenvalues. $\tilde{E}$ should lie in one of them. By Theorem \ref{omega1}, $m(B)\leq Ce^{-(\eta_0-\delta_0)(2n+1)}$. So there is some e.v. $E_{j,[n+1,3n+1],\tilde{\omega}}$ of $H_{[n+1,3n+1],\omega}$ s.t.
   \[
   \vert\tilde{E}-E_{j,[n+1,3n+1],\tilde{\omega}}\vert\leq e^{-(\eta_0-\delta_0)(2n+1)}
   \]
  By the same argument, $\exists E_{i,[-n,n],\tilde{\omega}}$, s.t.
   \[
   \vert\tilde{E}-E_{i,[-n,n],\tilde{\omega}}\vert\leq e^{-(\eta_0-\delta_0)(2n+1)}
   \]
   \item So $\vert E_{i,[-n,n],\tilde{\omega}}-E_{j,[n+1,3n+1],\tilde{\omega}}\vert\leq 2e^{-(\eta_0-\delta_0)(2n+1)}$. However, by Theorem \ref{omega3}, one have $E_{j,[n+1,3n+1],\tilde{\omega}}\notin B_{[-n,n],\epsilon,\tilde{\omega}}$, while $E_{i,[-n,n],\tilde{\omega}}\in B_{[-n,n],\epsilon,\tilde{\omega}}$
   This will give us a contradiction below.
\end{itemize}
~\\
Since $\vert E_{i,[-n,n],\tilde{\omega}}-E_{j,[n+1,3n+1],\tilde{\omega}}\vert\leq 2e^{-(\eta_0-\delta_0)(2n+1)}$ and $E_{i,[-n,n],\tilde{\omega}}$ being the e.v. of $H_{[-n,n],\tilde{\omega}}$,
\[
  \left\Vert G_{[-n,n],E_{j,[n+1,3n+1],\tilde{\omega}},\tilde{\omega}}\right\Vert\geq \frac{1}{2}e^{(\eta_0-\delta_0)(2n+1)}
\]
So $\exists y_{n1},y_{n2}\in [-n,n]$ s.t.
NEED FIX
\[
  \left\vert G_{[-n,n],E_{j,[n+1,3n+1],\tilde{\omega}},\tilde{\omega}}(y_{n1},y_{n2})\right\vert\geq \frac{1}{2}e^{(\eta_0-\delta_0)(2n+1)}
\]
But $E_{j,[n+1,3n+1],\omega}\notin B_{[-n,n],\epsilon,\tilde{\omega}}$, i.e.
\[
\vert P_{[-n,n],\epsilon,E_{j,[n+1,3n+1],\omega},\tilde{\omega}}\vert\geq e^{(\gamma(E_j)-\epsilon)(2n+1)}
\]
so
\begin{equation}\label{last}
  \left\Vert P_{[-n,y_{n1}],\epsilon,E_j}P_{[y_{n2},n],\epsilon,E_j}\right\Vert\geq\frac{1}{2}e^{(\eta_0-\delta-0)(2n+1)}e^{(\gamma(E_j)-\epsilon)(2n+1)}
\end{equation}
Let $M= sup\{|V|+|E_i|+|E_j|+2\}$, where $|V|$ is assumed bounded, $E_i,E_j$ are bounded because they are close to $E\in I$.\\
Then pick $\epsilon$ small enough in Theorem \ref{omega3} s.t.
  \begin{equation}\label{epsilon1}
    2\epsilon<\min\{\eta_0-\delta_0,\nu\}
  \end{equation}
and fix it, then let
  \[L:=e^{(\eta-\delta)}e^{(\nu-\epsilon)}>1\]
Pick $K$ big enough in Theorem \ref{omega3} to be s.t.
  \[(3M)^{\frac{1}{K}}<L\]
say, $\exists \sigma>0$,
\begin{equation}\label{K}
(3M)^{\frac{1}{K}}\leq L-\sigma<L
\end{equation}
then for left hand side of \eqref{last}, there are three cases:
\begin{enumerate}
  \item both $|-n-y_{1n}|>\frac{n}{K}$ and $|n-y_{2n}|>\frac{n}{K}$
  \item one of them is large, say $|-n-y_{1n}|>\frac{n}{K}$ and $|x_{2n}-y{_2n}|\leq\frac{n}{K}$
  \item both small.
\end{enumerate}

for $(1)$,
\[
\frac{1}{2}e^{(\eta_0-\delta_0+\gamma(E_j)-\epsilon)(2n+1)}\leq e^{2n(\gamma(E_j)+\epsilon)}
\]
by our choice \eqref{epsilon1},
 $\eta-\delta+\gamma(E_j)-\epsilon>\gamma(E_j)+\epsilon$. Then for $n$ large enough, we get contradiction.

for $(2)$, similarly with \eqref{epsilon1} and\eqref{K}
\[
  \begin{aligned}
    \frac{1}{2}e^{(\eta_0-\delta_0+\gamma(E_j)-\epsilon)(2n+1)}
    &\leq e^{(\gamma(E_j)+\epsilon)(n+1)}(3M)^{\frac{n}{K}}\\
    &\leq e^{(\gamma(E_j)+\epsilon)(n+1)} L^n\\
    &\leq e^{(\gamma(E_j)+\epsilon)(n+1)} e^{(\eta_0-\delta_0+\gamma(E_j)-\epsilon)n}\\
    \frac{1}{2}e^{(\eta_0-\delta_0+\gamma(E_j)-\epsilon)(n+1)}
    &\leq e^{(\gamma(E_j)+\epsilon)(n+1)}
  \end{aligned}
\]
We get contradiction.

for $(3)$,with \eqref{epsilon1} and \eqref{K}
\[
\begin{aligned}
  \frac{1}{2}e^{(\eta_0-\delta_0+\gamma(E_j)-\epsilon)(2n+1)}
  &\leq (3M)^{\frac{2n}{K}}\\
  &\leq (L-\sigma)^{2n}\\
  &\leq(e^{(\eta_0-\delta_0+\gamma(E_j)-\epsilon)}-\sigma)^{2n}
\end{aligned}
\]
Contradiction.

So our assumption that $2n+1$ is not eventually $(\gamma(\tilde{E})-8\epsilon_0,n,\tilde{E},\tilde{\omega})$-regular is flase. Theorem \ref{thm2} follows.
\end{proof}
