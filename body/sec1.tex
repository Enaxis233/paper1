\section{Introduction}
The Anderson model is given by a class of discrete analogs of Schr$\ddot{o}$dinger operators $H_{\omega}$ with real $i.i.d$ potetials $\{V_{\omega}(n)\}$:
\begin{equation}
  (H_{\omega}\Psi)(n)=\Psi(n+1)+\Psi(n-1)+V_{\omega}(n)\Psi(n),
\end{equation}
where $\omega=\{\omega_n\}_{n\in\mathbb{Z}}\in\Omega=S^{\mathbb{Z}}$, $S=supp\{\mu\}\subset \mathbb{R}$ is assumed to be compact and contains at least two points, $\mu$ is a borel probability on $\mathbb{R}$. $i.e.$ for each $n\in\mathbb{Z}$, $V_{\omega} (n)$ is $i.i.d.$ random variables depending on $\omega_n$ in $(S,\mu)$, but we will consider $V_\omega$ in the product probability space $ (S^{\mathbb{Z}},\mu^{\mathbb{Z}})$ as a whole instead.

We say that $H_\omega$ exhibits the pectral localization property in an interval $I$ if for $a.e.\omega$, $H_\omega$ has only pure point spectrum in $I$ and its eigenfunction $\Psi(n)$ decays exponentially in $n$. We are gonna give a new proof for Anderson model based on the large deviation estimates and subharmonicity of Lvapunov exponents.
