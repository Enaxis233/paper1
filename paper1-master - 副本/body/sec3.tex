\section{main technique}
% The main technique is to find the proper set $\Omega_0$. Roughly speaking, the idea is: a point $x$ is $(C,k,E,\omega)$-singular means the corresponding pair $(E,\omega)$ is in some "bad" "large deviation set" for operators restricted near $x$, say, $H_{[x-k,x+k]}$. We can then pick proper set $\Omega_1$ such that these bad sets have small measures. Then pick $\Omega_2$ such that $E$ is not only in these bad sets, but also stay very close to eigenvalues of $H_{[x-k,x+k]}$ which are also in these bad sets. In this case, if we pick $\tilde{\omega}\in\Omega_1\cap\Omega_2$ and if $0$ and $2n+1$ are both $(C,n,\tilde{E},\tilde{\omega})$-singular, then $\tilde{E}$ will  be close to both eigenvalues $E_{i,n}$ of $H_{[-n,n]}$ near $0$ and eigenvalues $E_{j,n}$ of $H_{[n+1,3n+1]}$ near $2n+1$. So now $\tilde{E}$ is in bad sets for both intervals and are closed to both e.v., where each e.v. only belongs to bad sets for their own intervals. Which leads to a contradiction because the bad sets are so small that we can pick $\Omega_3$ by Borel Cantelli, such that eventually all the eigenvalues of $[n+1,3n+1]$ can't stay in bad set for $[-n,n]$,

% The main idea is that a point $x$ is $(C,k,E,\omega)$-singular means somehow the pair $(E,\omega)$ is in some "bad" "large deviation set" for operators restricted near $x$, say, $H_{[x-k,x+k]}$.(see lemma1) However we can get from large deviation theorem that these sets are small, so both $E$ and eigenvalues of $H_{[x-k,x+k]}$, which also lies in these bad sets, will eventually get closed to each other due to some nice properties of these sets.

We introduce the large deviation theorem here without proof.
\cite{tsay1999some}
\begin{lemma}[large deviation estimates]
  For any $\epsilon>0$, there exists $\eta=\eta(\epsilon)>0$ such that, $\exists N_0$, $ \forall b-a>N_0$
  \[
  \mu \left\{ \omega:\left\vert \frac{1}{b-a+1} ln\Vert P_{[a,b],E,\omega}\Vert-\gamma(E) \right\vert\geq\epsilon
   \right\} \leq e^{-\eta (b-a+1)}
  \]
\end{lemma}
\begin{remark}
  If we denote
\[
\begin{split}
    B_{[a,b],\epsilon}=&\left\{(E,\omega): |P_{[a,b],E,\omega}|\geq e^{(\gamma(E)+\epsilon)(b-a+1)}\right\} \\
    &\bigcup \left\{(E,\omega): |P_{[a,b],E,\omega}|\leq e^{(\gamma(E)-\epsilon)(b-a+1)}\right\}
\end{split}
\]
  we find that
  \[
    B_{[a,b],\epsilon}\subseteq\left\{(E,\omega):\left\vert\frac{1}{b-a+1} ln\Vert P_{[a,b],E,\omega}\Vert-\gamma(E)\right\vert \geq\epsilon \right\}
  \]
  This is the "bad" set. And Large deviation theorem gives us the estimates that for all $E,a,b$
\begin{equation}\label{ldt}
  P(\{\omega:(E,\omega)\in B_{[a,b],\epsilon}\})\leq e^{-\eta(b-a+1)}
\end{equation}
Moreover, we denote
\begin{equation}\label{B+}
    B_{[a,b],\epsilon}^{+} =\left\{(E,\omega): |P_{[a,b],E,\omega}|\geq e^{(\gamma(E)+\epsilon)(b-a+1)}\right\}
\end{equation}
\begin{equation}\label{B-}
    B_{[a,b],\epsilon}^{-} =\left\{(E,\omega): |P_{[a,b],E,\omega}|\leq e^{(\gamma(E)-\epsilon)(b-a+1)}\right\}\
\end{equation}
and denote $B_{[a,b],\epsilon,E}=\{\omega:(E,\omega)\in B_{[a,b]}\}$. Others similar.
\end{remark}

Assume $\epsilon=\epsilon_0<\frac{1}{8}\nu$ is fixed for now,s.t. so we omit it from $B_{[a,b]}$ untill Theorem \ref{omega3}. And $\eta_0$ is the corresponding parameter
\begin{lemma}\label{lemma1}
 $n \geq 2$, if $x$ is $(\gamma(E)-8\epsilon_0,n,E,\omega)$-singular, then $(E,\omega)\in B_{[x-n,x+n]}^-\cup B_{[x-n,x]}^+\cup B_{[x,x+n]}^+$.
\end{lemma}
\begin{remark}
  Note that from \eqref{ldt}, for all $E,x,n\geq 2$,
  \[
    P(B_{[x-n,x+n],E}^-\cup B_{[x-n,x],E}^+\cup B_{[x,x+n],E}^+)\leq 3Ce^{-\eta_0 n}
  \]
\end{remark}
\begin{proof}
Assume not, then
\[
  \left\{
  \begin{aligned}
     & |P_{[x-n,x+n],E,\omega}|\geq e^{(\gamma(E)+\epsilon_0)(2n+1)} \\
     & |P_{[x-n,x],E,\omega}|\leq e^{(\gamma(E)-\epsilon_0)(n+1)} \\
     & |P_{[x,x+n],E,\omega}|\leq e^{(\gamma(E)-\epsilon_0)(n+1)}
  \end{aligned}
  \right.
\]
So we can estimate
\[
  \begin{aligned}
    \left\vert G_{[x-n,x+n],E,\omega}(x,x-n)\right\vert
    &=  \frac{\left\vert P_{[x,x+n],E,\omega}\right\vert}{\left\vert P_{[x-n,x+n],E,\omega}\right\vert}\\
    &\leq  \frac{e^{(\gamma(E)+\epsilon_0)(n+1)}}{e^{(\gamma(E)-\epsilon_0)(2n+1)}}\\
    &\leq  e^{-\gamma(E)(n)+\epsilon_0(3n+2)}\\
    &\leq  e^{-(\gamma(E)-8\epsilon_0)n}
  \end{aligned}
\]
Similar for $G_{[x-n,x+n],E,\omega}(x,x+n)$.
Thus $x$ is $(\gamma(E))-8\epsilon_0, n, E,\omega)$-regular, contradiction.
\end{proof}

By Theorem \ref{thm2},
% From lemma \ref{lemma1}, we get our first restriction on $\Omega$ to make the bad sets have small measure.

\begin{thm}\label{omega1}
  Let $0<\delta_0<\eta_0$, Let $E\in I$. For a.e. $\omega$ (denote as $\Omega_1$), $\exists N_1=N_1(\omega)$, s.t. $\forall n>N_1$, $m(B_{[n+1,3n+1],\omega}^-)\leq e^{-(\eta_0-\delta_0)(2n+1)}$ and $m(B_{[-n,n],\omega}^-)\leq e^{-(\eta_0-\delta_0)(2n+1)}$
\end{thm}

\begin{proof}
  We only prove for $m(B_{[n+1,3n+1],\omega}^-)$.

  By \eqref{ldt}, $\forall E\in I$, $P(B_{[n+1,3n+1],E}^-)\leq Ce^{-\eta_0(2n+1)}$.

  If we denote
    \[
      \Omega_{\delta_0,n}=\left\{\omega:m(B_{[n+1,3n+1],\omega}^-)\leq e^{-(\eta_0-\delta_0)(2n+1)}\right\}
    \]

  By chebyshev,
  \[
    \begin{aligned}
      P(\Omega_{\delta_0,n}^c)
      &\leq e^{(\eta_0-\delta_0)n}\int_{\Omega} m(B_{[n+1,3n+1],\omega})dP\omega\\
      &=    e^{(\eta_0-\delta_0)n}\int_I P(B_{[n+1,3n+1],E})dx\\
      &\leq e^{(\eta_0-\delta_0)n}m(I)e^{-\eta_0(2n+1)}\\
      &=m(I)e^{-\delta(2n+1)}
    \end{aligned}
  \]
By Borel-Cantelli lemma, we get

for $a.e.~\omega$, $\exists N$, s.t. $\forall n>N$, $\omega\in\Omega_{\delta,n}$, i.e. $m(B_{[n+1,3n+1],\omega}^-)\leq e^{-(\eta_0-\delta_0)(2n+1)}.$
Similar for $[-n,n]$, pick insection for $\Omega_1$, and maximum for $N_1$.
\end{proof}

\begin{thm}[Craig-Simon]
  For a.e.$\omega$(denote as $\Omega_2$), we have
  \[
    \begin{aligned}
      &\varlimsup_{n\to\infty} \frac{1}{n+1} ln\Vert T_{[-n,0],E,\omega}\Vert\leq\gamma(E)\\
      &\varlimsup_{n\to\infty} \frac{1}{n+1} ln\Vert T_{[0,n],E,\omega}\Vert\leq\gamma(E)\\
      &\varlimsup_{n\to\infty} \frac{1}{n+1} ln\Vert T_{[n+1,2n+1],E,\omega}\Vert\leq\gamma(E)\\
      &\varlimsup_{n\to\infty} \frac{1}{n+1} ln\Vert T_{[2n+1,3n+1],E,\omega}\Vert\leq\gamma(E)
    \end{aligned}
  \]
\end{thm}
\begin{cor}\label{omega2}
  $\forall \omega\in\Omega_2$, $\exists N_2=N_2(\omega)$, s.t. $\forall n>N_2$,
  \[
    \begin{aligned}
      &\Vert T_{[-n,0],E,\omega}\Vert< e^{(\gamma(E)+\epsilon)(n+1)}\\
      &\Vert T_{[0,n],E,\omega}\Vert< e^{(\gamma(E)+\epsilon)(n+1)}\\
      &\Vert T_{[n+1,2n+1],E,\omega}\Vert< e^{(\gamma(E)+\epsilon)(n+1)}\\
      &\Vert T_{[2n+1,3n+1],E,\omega}\Vert< e^{(\gamma(E)+\epsilon)(n+1)}
    \end{aligned}
  \]
\end{cor}
\begin{remark}
  The only difference between here and \cite{craig1983subharmonicity} is the restrict intervals. As long as one follows the proof there, one can get the results here.
\end{remark}
% \begin{proof}
%   Only prove $[n+1,2n+1]$ case.\\
%   Claim:
%   \begin{enumerate}
%     \item $\gamma(E)$ is subharmonic.
%     \item $\varlimsup_{n\to\infty} \frac{1}{n+1} ln\Vert T_{[n+1,2n+1],E,\omega}\Vert$ is submean
%   \end{enumerate}
% \end{proof}
\begin{thm}\label{omega3}
   $\epsilon>0,K>0$, For a.e.$\omega$(denote as $\Omega_3=\Omega_3(\epsilon,K)$), $\exists N_3=N_3(\omega)$, $\forall n>N_3$, $\forall E_{j,[n+1,3n+1],\omega}$ being eigenvalue of $H_{[n+1,3n+1],\omega}$, $\forall y_1,y_2$ satisfy $-n\leq y_1\leq y_2\leq n$,  $\abs{-n-y_1}\geq\frac{n}{K}$, and $\abs{n-y_2}\geq\frac{n}{K}$,
 we have $E_{j,[n+1,3n+1],\omega}\notin B_{[-n,y_1],\epsilon,\omega}\cup B_{[y_2,n],\epsilon,\omega}$.
\end{thm}
\begin{remark}
  Note that $\epsilon$ and $K>0$ is not determined yet, we're gonna determine it later on in section 4.
\end{remark}

\begin{proof}
In order to use Borel-Cantelli, one need to estimate
\[
  P\left(\bigcup\limits_{y_1,y_2}\bigcup\limits_{j=1}^{2n+1}
  B_{[-n,y_1],\epsilon,E_{j,[n+1,3n+1],\omega}}\cup B_{[y_2,n],\epsilon,E_{j,[n+1,3n+1],\omega}}\right)
\]
where $y_1,y_2$ satisfy assumptions above. Denote it by $\bar{P}$. Consider
\[
  \begin{aligned}
    P\left(B_{[y_2,n],\epsilon,E_{j,[n+1,3n+1],\omega}}\right)
    &= \int_\Omega \chi_{B_{[y_2,n],\epsilon,E_{j,[n+1,3n+1],\omega}}} dP\omega\\
    &= \int_{S^{2n+1}}\left(\int_{\tilde{\Omega}} \chi_{B}~ d\tilde{\mu}\right)d\mu^{2n+1}(\omega_{n+1},\cdots,\omega_{3n+1})\\
    &=\int_{S^{2n+1}} \tilde{P}(\tilde{B}_{[y_2,n],\epsilon,E_{j,[n+1,3n+1],\omega}})d\mu^{2n+1}(\omega_{n+1},\cdots,\omega_{3n+1})
  \end{aligned}
\]
where for $\tilde{\Omega}$ and $\tilde{\mu}$, one take away the $[n+1,3n+1]$ terms from $\Omega$ and $\mu^{\mathbb{Z}}$
However, for any fixed $E$, $B_{[y_2,n],\epsilon,E}$ is of the form
\[
  \left(\bigotimes\limits_{i\in[y_2,n]}S\right) \times {B'_{[y_2,n],\epsilon}}
\]
where
\[
  B'_{[y_2,n],\epsilon,E}=\left\{\omega|_{[y_2,n]}:\omega\in B_{[y_2,n],\epsilon,E}\right\}
\]
So,
\[
  \begin{aligned}
    P(B_{[y_2,n],\epsilon,E})
    &= \int_{S^{2n+1}}\left(\int_{\tilde{\Omega}} \chi_{B_{[y_2,n],\epsilon,E}}~ d\tilde{\mu}\right)d\mu^{2n+1}(\omega_{n+1},\cdots,\omega_{3n+1})\\
    &=\tilde{P}(\tilde{B}_{[y_2,n],E})\times 1\times\cdots\times 1\\
    &=\tilde{P}(\tilde{B}_{[y_2,n],E})
  \end{aligned}
\]
Br \eqref{ldt},
\[
\tilde{P}(\tilde{B}_{[y_2,n],\epsilon,E})\leq Ce^{-\eta\abs{n-y_2}},\quad \forall E
\]
So
\[
P\left(B_{[y_2,n],\epsilon,E_{j,[n+1,3n+1],\omega}}\right)\leq Ce^{-\abs{n-y_2-}}\leq Ce^{-n/K}
\]
\[
\bar{P}\leq C(2n+1)^3e^{-n/K}
\]
The sum over n is finite, use Borel-Cantelli, we can get the result.
\end{proof}
